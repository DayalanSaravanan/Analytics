\documentclass[a4paper,11pt,openright]{report}
\setlength{\parindent}{0pt} % set noindent for entire file

\usepackage[utf8]{inputenc}
\usepackage[a4paper, left=10mm, right=10mm, top=20mm]{geometry}
\usepackage{xcolor,graphicx}
\usepackage{amsmath}
\usepackage{setspace}
\usepackage{sectsty}
\usepackage{etoolbox}
\usepackage{enumitem}
\usepackage{listings}
\usepackage{times}
%\usepackage{lmodern}

\graphicspath{ {/home/saran/Analytics/May_01/} }

\lstdefinestyle{mystyle}{
	backgroundcolor=\color{white},
	basicstyle=\ttfamily\footnotesize,
	breakatwhitespace=false,
	breaklines=true,
	captionpos=b,
	keepspaces=true,
	showspaces=false,
	showstringspaces=false,
	showtabs=false,
	tabsize=4
}

\lstset{style=mystyle}

\begin{document}
\singlespacing
\pagestyle{plain}

\begin{center}
\textbf{Assignment Probability} \\
Date: 04/05/2020 \hspace{2mm} Name: D.Saravanan
\end{center}

\vspace{10px}

\begin{enumerate}

\item[1.] Three coins are tossed. Find the probability of getting
\begin{enumerate}

\item[a)] At least one head \\
The outcomes of this experiment are ordered pairs of H and T. \\
The sample space is S = \{HHH, TTT, HTT, THT, TTH, THH, HTH, HHT\} \\
The sample space has eight total number of outcomes. \\
Let event E =  at least one head. \\
There are seven outcomes that meet this condition, 
 \{HHH, HTT, THT, TTH, THH, HTH, HHT\}.\\
\begin{equation*}
P(E) = \frac{7}{8} = 0.875	
\end{equation*}

\item[b)] Exactly 2 heads \\
Let event E = exactly 2 heads. \\
There are three outcomes that meet this condition,
\{THH, HTH, HHT\}. \\
\begin{equation*}
P(E) = \frac{3}{8} = 0.375
\end{equation*}

\end{enumerate}
\vspace{5px}


\vspace{10px}

\pagebreak

\item[2.] An integer is choosen at random out of the integers from 1 to 100. What is the
probability that it is

\begin{enumerate}

\item[a)] Multiple of 5 \\
The sample space is S = \{1, 2, 3, . . ., 100\} \\
From integers 1 to 100, there are 20 integers that are multiple by 5, which are known from
$100//5 = 20$. \\
Let event E = multiple of 5. 
\begin{equation*}
P(E) = \frac{20}{100} = 0.2
\end{equation*}

\item[b)] Divisible by 7 \\
From integers 1 to 100, there are 14 integers that are divisible by 7, which are known from
as $100//7 = 14$. \\
Let event E = divisible by 7.
\begin{equation*}
P(E) = \frac{14}{100} = 0.14
\end{equation*}

\item[c)] Greater then 70 \\
From integers 1 to 100, there are 30 integers that are greater than 70. \\
Let event E = greater then 70. \\
\begin{equation*}
P(E) = \frac{30}{100} = 0.3
\end{equation*}
\end{enumerate}


Program:

\vspace{5px}

\pagebreak

Output:

\vspace{10px}

\pagebreak

\item[3.] Generate Prime Numbers till 1000 and find the following among this population.
\begin{enumerate}
\item[a)] Mean
\item[b)] Median
\item[c)] Range
\item[d)] Quartile
\item[e)] Inter-Quartile Range
\item[f)] Variance
\item[g)] Standard Deviation
\end{enumerate}

Program: 
\lstinputlisting[language=Python]{prime.py}

\vspace{5px}

\pagebreak

Output:

\end{enumerate}

\end{document}
