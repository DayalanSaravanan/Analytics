\documentclass[a4paper,11pt,openright]{report}
\setlength{\parindent}{0pt} % set noindent for entire file

\usepackage[utf8]{inputenc}
\usepackage[a4paper, left=10mm, right=10mm, top=20mm]{geometry}
\usepackage{xcolor,graphicx}
\usepackage{amsmath}
\usepackage{setspace}
\usepackage{sectsty}
\usepackage{etoolbox}
\usepackage{enumitem}
\usepackage{listings}
\usepackage{times}
%\usepackage{lmodern}

\graphicspath{ {/home/saran/Analytics/May_01/} }

\lstdefinestyle{mystyle}{
	backgroundcolor=\color{white},
	basicstyle=\ttfamily\footnotesize,
	breakatwhitespace=false,
	breaklines=true,
	captionpos=b,
	keepspaces=true,
	showspaces=false,
	showstringspaces=false,
	showtabs=false,
	tabsize=4
}

\lstset{style=mystyle}

\begin{document}
\singlespacing
\pagestyle{plain}

\begin{center}
\textbf{Assignment Poisson Distribution} \\
Date: 04/05/2020 \hspace{2mm} Name: D.Saravanan
\end{center}

\vspace{10px}

\begin{enumerate}

\item[1.] Find the probability that atmost $5$ defective fuses will be found in a box of 
$200$ fuses if experience shows that $2$ per cent of such fuses are defective. \\

\begin{enumerate}

\item[a)] At least $3$ successes \\
The outcomes of this experiment are ordered pairs of H and T. \\
The sample space is S = \{HHH, TTT, HTT, THT, TTH, THH, HTH, HHT\} \\
The sample space has eight total number of outcomes. \\
Let event E =  at least one head. \\
There are seven outcomes that meet this condition, 
 \{HHH, HTT, THT, TTH, THH, HTH, HHT\}.\\
\begin{equation*}
P(E) = \frac{7}{8} = 0.875	
\end{equation*}

\item[b)] At most $3$ successes \\
Let event E = exactly 2 heads. \\
There are three outcomes that meet this condition,
\{THH, HTH, HHT\}. \\
\begin{equation*}
P(E) = \frac{3}{8} = 0.375
\end{equation*}

\item[c)] Exactly $3$ failures \\

\end{enumerate}

\item[2.] The number of accidents in a year attributed to taxi drivers in a city follows a
Poisson distribution with mean equal to $3$. Out of $1000$ taxi drivers, find approximetly
the number of drivers with \\

\begin{enumerate}

\item[a)] No accidents in a year \\
The sample space is S = \{1, 2, 3, . . ., 100\} \\
From integers 1 to 100, there are 20 integers that are multiple by 5, which are known from
$100//5 = 20$. \\
Let event E = multiple of 5. 
\begin{equation*}
P(E) = \frac{20}{100} = 0.2
\end{equation*}

\item[b)] More than $3$ accidents in a year \\
From integers 1 to 100, there are 14 integers that are divisible by 7, which are known from
as $100//7 = 14$. \\
Let event E = divisible by 7.
\begin{equation*}
P(E) = \frac{14}{100} = 0.14
\end{equation*}

\end{enumerate}

\item[3.] From the records of $10$ Indian Army corps kept over $20$ years the following data
were obtained showing the number of deaths caused by the horse. Calculate the theoretical
Poisson frequencies \\

\begin{enumerate}

\item[a)] Find the Frequencies of the distribution of heads and tabulate the results.
\item[b)] Calculate the mean number of sucess and standard deviations.

\end{enumerate}

\end{enumerate}
\end{document}
