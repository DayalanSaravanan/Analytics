\documentclass[a4paper,10pt,openright]{report}
\setlength{\parindent}{0pt} % set noindent for entire file

\usepackage[utf8]{inputenc}
\usepackage[a4paper, top=2cm, left=1cm, right=1.5cm]{geometry}
\usepackage{xcolor,graphicx}
\usepackage{amsmath}
\usepackage{setspace}
\usepackage{sectsty}
\usepackage{etoolbox}
\usepackage{enumitem}
\usepackage{listings}
\usepackage{times}

\graphicspath{ {/home/saran/Analytics/May_04/} }

\lstdefinestyle{mystyle}{
	backgroundcolor=\color{white},
	basicstyle=\ttfamily\footnotesize,
	breakatwhitespace=false,
	breaklines=true,
	captionpos=b,
	keepspaces=true,
	showspaces=false,
	showstringspaces=false,
	showtabs=false,
	tabsize=4
}

\lstset{style=mystyle}

\begin{document}
\singlespacing
\pagestyle{plain}

\begin{center}
\textbf{Assignment T-test} \\
Date: 05/05/2020 \hspace{2mm} Name: D.Saravanan
\end{center}

\vspace{10px}

\begin{enumerate}

\item[1.] Two sets of ten students selected at random from a college were taken. One set was
given memory test as they were and the other was given the memory test after two weeks of
training and the scores are given below. \\
\begin{tabular}{lrrrrrrrrrr}
Set A: & 10 & 8 & 7 & 9 & 8 & 10 & 9 & 6 & 7 & 8 \\ 
Set B: & 12 & 8 & 8 & 10 & 8 & 11 & 9 & 8 & 9 & 9 \\
\end{tabular} \\
Do you think there is a significant effect due to training?

\item[2.] A group of 5 patients treated with medicine A weighs (in kg) 42, 29, 48, 60 and 41
; a second group of 7 patients from the same hospital treated with medicine B weighs (in kg)
38, 42, 56, 64, 68, 69 and 62. Do you agree with the claim that medicine B increases weight
significantly. \\

\item[3.] Samples of two types of electric bulbs were tested for length of life and the
following data were obtained \\
\begin{tabular}{lrr}
		& Type I & Type II \\
No. of Samples: & 8 & 7 \\
Mean (hours):   & 1134 & 1024 \\
SD (hours):     & 35 & 40 \\
\end{tabular} \\
Test at 5 percent level, whether the difference in sample mean is significant.

\end{enumerate}
\end{document}
