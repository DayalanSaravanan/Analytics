\documentclass[a4paper,11pt,openright]{report}
\setlength{\parindent}{0pt} % set noindent for entire file

\usepackage[utf8]{inputenc}
\usepackage[a4paper, left=15mm, right=15mm, top=15mm]{geometry}
\usepackage{xcolor,graphicx}
\usepackage{amsmath}
\usepackage{setspace}
\usepackage{sectsty}
\usepackage{etoolbox}
\usepackage{enumitem}
\usepackage{listings}
\usepackage{times}

\graphicspath{ {/home/saran/Analytics/Apr_27/} }

\lstdefinestyle{mystyle}{
	backgroundcolor=\color{white},
	basicstyle=\ttfamily\footnotesize,
	breakatwhitespace=false,
	breaklines=true,
	captionpos=b,
	keepspaces=true,
	showspaces=false,
	showstringspaces=false,
	showtabs=false,
	tabsize=4
}

\lstset{style=mystyle}

\begin{document}
\singlespacing
\pagestyle{plain}

\begin{center}
\textbf{Answers to Question Set 7} \\
Date: 28/04/2020 \hspace{2mm} Name: D.Saravanan
\end{center}

\begin{enumerate}

\item[1.] Print the total missing values in the dataset. \\

Program:
\lstinputlisting[language=Python]{question1.py}

\vspace{5px}

Output:
\lstinputlisting{output1.txt}

\vspace{5px}

\item[2.] Print the percentage of missing values in the dataset column wise. \\

Program:
\lstinputlisting[language=Python]{question2.py}

\vspace{5px}

Output:
\lstinputlisting{output2.txt}

\vspace{5px}

\item[3.] Print the percentage of total unit missing values. \\

Program:
\lstinputlisting[language=Python]{question3.py}

\vspace{5px}

Output:
\lstinputlisting{output3.txt}

\vspace{5px}

\item[4.] Remove all rows which are having at the least one null values in it, what is the
total number of rows after dropping and what is the percentage of rows lost due to this 
operation. \\

Program:
\lstinputlisting[language=Python]{question4.py}

\vspace{5px}

Output:
\lstinputlisting{output4.txt}

\vspace{5px}

\item[5.] Remove all columns which are having at the least one null values in it. What is 
the total number of columns after dropping and what is the percentage of columns lost due to
this operation. \\

Program:
\lstinputlisting[language=Python]{question5.py}

\vspace{5px}

Output:
\lstinputlisting{output5.txt}

\vspace{5px}

\item[6.] Print all the columns name separated by a line. \\

Program:
\lstinputlisting[language=Python]{question6.py}

\vspace{5px}

Output:
\lstinputlisting{output6.txt}

\vspace{5px}

\pagebreak

\item[7.] Extract a sample of 100 rows at random from the dataset. \\

Program:
\lstinputlisting[language=Python]{question7.py}

\vspace{5px}

Output:
\lstinputlisting{output7.txt}

\vspace{5px}

\item[8.] Replace all the NA/Null values in the previous ouput (sample dataset) with the 
values that came directly after it in the same column. \\

Program:
\lstinputlisting[language=Python]{question8.py}

\vspace{5px}

Output:
\lstinputlisting{output8.txt}

\vspace{5px}

\item[9.] In continuation with the previous step fill the remaining missing values with
zeros. \\

Program:
\lstinputlisting[language=Python]{question9.py}

\vspace{5px}

Output:
\lstinputlisting{output9.txt}

\vspace{5px}

\item[10.] Print the missing values count of the cleaned sample dataset. \\ 	

Program:
\lstinputlisting[language=Python]{question10.py}

\vspace{5px}

Output:
\lstinputlisting{output10.txt}

\end{enumerate}

\end{document}
