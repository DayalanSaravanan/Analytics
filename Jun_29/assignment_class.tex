\documentclass[a4paper,11pt,openright]{report}
\setlength{\parindent}{0pt} % set noindent for entire file

\usepackage[utf8]{inputenc}
\usepackage[a4paper,top=20mm,left=10mm,right=10mm]{geometry}
\usepackage{xcolor,graphicx}
\usepackage{amsmath}
\usepackage{setspace}
\usepackage{sectsty}
\usepackage{etoolbox}
\usepackage{enumitem}
\usepackage{listings}
\usepackage{textcmds}
\usepackage{times}

\graphicspath{ {/home/saran/Analytics/Jun_29/} }

\lstdefinestyle{mystyle}{
	backgroundcolor=\color{white},
	basicstyle=\ttfamily\footnotesize,
	breakatwhitespace=false,
	breaklines=true,
	captionpos=b,
	keepspaces=true,
	showspaces=false,
	showstringspaces=false,
	showtabs=false,
	tabsize=4
}

\lstset{style=mystyle}

\begin{document}
\singlespacing
\pagestyle{plain}

\begin{center}
\textbf{Assignment Class} \\
Date: 29/06/2020 \hspace{2mm} Name: D.Saravanan
\end{center}

\vspace{10px}

\begin{enumerate}

\item[1.] Create a JAVA package containing basic operation to operate on two integers.

\vspace{0.5cm}

Program:
\lstinputlisting[language=Java]{operator.java}

\pagebreak

\item[2.] Write a JAVA program to import the package, get two numbers and operation from the
user and perform the same.

\vspace{0.5cm}

Program:
\lstinputlisting[language=Java]{program.java}

\pagebreak

Output:
\lstinputlisting{output.txt}

\end{enumerate}
\end{document}
